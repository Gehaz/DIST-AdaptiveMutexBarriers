\section{Discussion}
\label{sec:discussion}

We establish a time complexity separation between adaptive and non-adaptive implementations of mutual-exclusion locks, counters, stacks and queues, thus capturing an inherent cost incurred by adaptive algorithms in the TSO model.

This separation follows from a tradeoff that we prove between fence complexity and the growth rate of adaptivity functions. Specifically, we prove that the fence complexity of any read/write $n$-process algorithm with a linear (or sub-linear) adaptivity function is $\Omega(\log \log n)$. Our results apply for both the cache-coherent (CC) and the distributed shared-memory (DSM) models.

A corollary of our tradeoff is that constant fence-complexity adaptive implementations for these objects do not exist. Moreover,  the impossibility result holds regardless of the RMR complexity of the algorithm.
Following~\cite{DBLP:conf/podc/AttiyaHL13,DBLP:journals/dc/GolabHHW12},
our tradeoff applies also to algorithms that may use
comparison primitives, such as \emph{compare-and-swap} (CAS),
in addition to reads and writes.

Kim and Anderson presented an adaptive mutual exclusion algorithm whose RMR complexity is $O\big(min(k,\log n)\big)$, where $k$ is point contention \cite{DBLP:journals/dc/KimA07}, hence it is $f$-adaptive for a linear $f$. The fence complexity of their algorithm is logarithmic. However, our tradeoff only implies $\log \log n$ fence complexity (see Corollary \ref{corollary:linear}). Finding the tight tradeoff between fence complexity and the adaptivity function growth rate is an interesting research direction.

The memory model considered by this work is TSO. We remind the reader that TSO ensures that writes are not reordered, but it is possible to perform a read from address $a$ before a write to address $b \neq a$ that is earlier in program order is performed.
The \emph{partial store ordering} (PSO) model, supported by older SPARC, is weaker than TSO, as it also allows the reordering of writes to different locations. 

Recent work by Attiya, Hendler and Woelfel~\cite{PSOLowerBound} showed that one cannot win on both the fence and RMR complexities of read/write PSO algorithms for many fundamental objects,
including locks, counters and queues. They proved the following lower bound: let $f$ and $r$ respectively denote the numbers of fences and RMRs performed in an operation on such an object, then
\begin{equation}\label{equation:X}
  f \cdot \log \frac{r}{f} + 1\in \Omega(\log n).
\end{equation} 

\noindent They also showed that the bound is tight.

Attiya et al. \cite{DBLP:conf/podc/AttiyaHL13} presented a TSO read/write mutual exclusion algorithm where each passage incurs a logarithmic number of RMRs and a constant number of fences. Inequality \ref{equation:X} establishes a complexity separation between the TSO and PSO models, since it follows from it that no such algorithm exists for the PSO model. Another interesting research direction is to find a tight tradeoff between the RMR-complexity and fence-complexity of adaptive PSO algorithms.
