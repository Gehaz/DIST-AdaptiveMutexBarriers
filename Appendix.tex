
\section{Proofs Omitted from Paper Body}

\begin{claim-repeat} {claim: maintain-criticalness}
		Let $E$ be an execution fragment, and $e \in E$ an event issued by some process $p$.
		\begin{itemize}[$\bullet$]
			\item Assume $e$ is not critical in $E$, and let $F$ be an execution fragment such that $F \preceq E$ and $F \mid p = E \mid p$. Then $e$ is not critical in $F$.
			\item Assume $e$ is critical in $E$, and let $F$ be an execution fragment such that $E \preceq F$ and $F \mid p = E \mid p$. Then $e$ is critical in $F$.
		\end{itemize}
\end{claim-repeat}


\begin{proof} \mbox{}

	$\bullet$ Assume $e$ is not critical in $E$. If $e$ is either a fence or a transition event, then so is the case in $F$ and we done. Thus, assume $e$ is either a read or a write event. If $e$ is a local event in $E$ then $e$ is a local event in $F$ and the claim clearly holds. Otherwise, $e$ is a remote event in both $E$ and $F$. The following two cases exists:
	\begin{itemize}
	\item [$e = read(v)$.] Since $e$ is not critical in $E$, $e$ is not the first remote read of $v$ by $p$ in $E$, and since $F \mid p = E \mid p$, $e$ is not the first remote read of $v$ by $p$ in $F$, thus $e$ is a not critical in $F$.
	\item [$e = write(v)$.] Since $e$ is not critical in $E$, $p$ is the last process to commit a write to $v$ before $e$ in $E$. Denote this write event by $e'$. Since $F \mid p = E \mid p$, $e'$ appears in $F$ as well. There is no write commit between $e'$ and $e$ in $E$, and since $F \preceq E$ there is no write commit between $e'$ and $e$ in $F$ as well. Therefore $p$ is the last process to commit a write to $v$ before $e$ in $F$, thus $e$ is not critical in $F$.
	\end{itemize}
	
	$\bullet$ Assume $e$ is a critical event in $E$. Then $e$ is either a critical read or write event in $E$. We consider each of the cases:
	\begin{itemize}
	\item [$e = read(v)$.] Then $e$ is the first remote read of $v$ by $p$ in $E$. Since $F \mid p = E \mid p$, $e$ is the first remote read of $v$ by $p$ in $F$, thus a critical read in $F$.
	\item [$e = write(v)$.] Thus, the last process to commit a write to $v$ before $e$ in $E$ (if any) is not $p$. Since $F \mid p = E \mid p$, no write commit by $p$ has been added to $F$, and since $E \preceq F$, no write commit by another process has been removed, thus the last process to commit a write to $v$ before $e$ in $F$ (if any) is not $p$, and $e$ is a critical write in $F$.
	\end{itemize}
\end{proof}




\begin{lemma-repeat} {lemma: erase-invis-procs}
	Let $E$ be an execution, $\mathit{INV}$ be an IN-set of $E$ and $Y \subseteq \mathit{INV}$.
	\newline Define $F = E^{-Y}$. Then the following hold:
	\begin{enumerate}
		\item $F$ is an execution;
		\item $Act(F) = Act(E) \setminus Y$ and $Fin(F) = Fin(E)$;
		\item $\mathit{INV} \setminus Y$ is an IN-set of $F$;
		\item Each $p \in Act(F)$ executes the same critical events in $F$ and in $E$.
	\end{enumerate}
\end{lemma-repeat}


\begin{proof}  \mbox{}
	
	We first prove the claim for the case $Y = \{p\}$, a single process.
	
	1. The ability to erase $p$ from $E$ and get an execution follows from \ref{IN-set:AW}, as there is no other process in the system that is aware of $p$. This implies that $p$ never writes to a variable that some other process read, and hence removing the events of $p$ does not effect the events of other processes. More formally, the following claim can be proven easily by induction on the length of $E$:
	Let $E$ be an execution and let $p \in P$ be a process such that $p \notin AW(q,E)$ for any $q \neq p$. Then $E^{-p}$ is an execution.
	
	2. We remove the events of $p \in Act(E)$ only, while the rest of the processes are executing the same events in both $F$ and $E$. Therefore $Act(F) = Act(E) \setminus \{p\}$ and $Fin(F) = Fin(E)$.
	
	3. We now prove $INV \setminus \{p\}$ is an IN-set of $F$.
	\\ \ref{IN-set:AW}: Consider $q \neq p$. Since $E \mid q = F \mid q$, we have $AW(q,F) = AW(q,E)$. By \ref{IN-set:AW}, $AW(q,E) \cap INV \subseteq \{q\}$, in particular $AW(q,F) \cap (INV \setminus \{p\}) \subseteq \{q\}$. $p$ executes no events in $F$, thus $AW(p,F) = \emptyset$ and \ref{IN-set:AW} holds for $p$ as well.
	\\ \ref{IN-set:entry-status}: For any $q \in INV \setminus \{p\}$, we have $F \mid q = E \mid q$, thus $status(q,F) = status(q,E) = entry$.	
	\\ \ref{IN-set:erase-processes}: Consider $Z \subseteq INV \setminus \{p\}$, and $e \in F^{-Z} = E^{-(Z \cup \{p\})}$. Notice that $e \in E, E^{-p}, F^{-(Z \cup \{p\})}$.  Since $Z, \{p\} \subseteq INV$ and $INV$ is an IN-set of $E$, by \ref{IN-set:erase-processes} applied to $E$ we have: $e$ is a critical event in $E^{-Z \cup \{p\}} = F^{-Z}$ if and only if $e$ is a critical event in $E$ if and only if $e$ is a critical event in $E^{-p} = F$.
	\\ \ref{IN-set:owner}: Consider an event $e \in F$ by some process $q$ accessing a remote variable $v$. Since $e \in E$ also, by \ref{IN-set:owner} $owner(v) \notin Act(E)$, and thus $owner(v) \notin Act(F) \subseteq Act(E)$.
	\\ \ref{IN-set:last-writer}: Assume $|Accessed(v,F) \cap Act(F)| > 1$ for some $v$. Since $F \preceq E$ and $Act(F) \subseteq Act(E)$ we get $|Accessed(v,E) \cap Act(E)| > 1$, and by \ref{IN-set:last-writer} $writer(v,E) \notin INV$. The only events removed are by $p \in INV$, and in particular we do not remove the events of $writer(v,E)$. Hence $writer(v,F) = writer(v,E) \notin INV$, and as so $writer(v,F) \notin INV \setminus \{p\}$.
	
	4. Follows directly from \ref{IN-set:owner} for $E$ and the fact that $p \in INV$.
	

	For the general case we prove the claim by induction on $|Y|$. The base case $|Y| = 1$ has been proven above. Assume we proved the claim for any $|Y|=n$, and consider $Y \subseteq INV$ such that $|Y| = n+1$. Fix an arbitrary $q \in Y$, and denote $Z = Y \setminus \{q\}$. Then $Z \subseteq INV$ and $|Z|=n$. Denote $F_Z = E^{-Z}$, then by induction hypothesis:
	\begin{enumerate}
		\item $F_Z$ is an execution;
		\item $Act(F_Z) = Act(E) \setminus Z$ and $Fin(F_Z) = Fin(E)$;
		\item $INV \setminus Z$ is an IN-set of $F_Z$;
		\item Each $p \in Act(F_Z)$ executes the same critical events in $F_Z$ and in $E$;
	\end{enumerate}
	Notice that $F = F_Z^{-q}$ and $q \in INV \setminus Z$, thus using the induction base with $F_Z$ and $\{q\}$ we get:
	\begin{enumerate}
		\item $F$ is an execution;
		\item $Act(F) = Act(F_Z) \setminus \{q\} = Act(E) \setminus Y$, and $Fin(F) = Fin(F_Z) = Fin(E)$;
		\item $(INV \setminus Z) \setminus \{q\} = INV \setminus Y$ is an IN-set of $F$;
		\item Each $p \in Act(F)$ executes the same critical events in $F$ and in $F_Z$, and thus the same critical events in $F$ and in $E$.
	\end{enumerate}
\end{proof}

