
\section{Proofs Omitted from Paper Body}

\begin{claim-repeat} {claim: maintain-criticalness}
		Let $E$ be an execution fragment, and $e \in E$ an event issued by some process $p$.
		\begin{itemize}[$\bullet$]
			\item Assume $e$ is not critical in $E$, and let $F$ be an execution fragment such that $F \preceq E$ and $F \mid p = E \mid p$. Then $e$ is not critical in $F$.
			\item Assume $e$ is critical in $E$, and let $F$ be an execution fragment such that $E \preceq F$ and $F \mid p = E \mid p$. Then $e$ is critical in $F$.
		\end{itemize}
\end{claim-repeat}


\begin{proof} \mbox{}

	$\bullet$ Assume $e$ is not critical in $E$. If $e$ is either a fence or a transition event, then so is the case in $F$ and we done. Thus, assume $e$ is either a read or a write event. If $e$ is a local event in $E$ then $e$ is a local event in $F$ and the claim clearly holds. Otherwise, $e$ is a remote event in both $E$ and $F$. The following two cases exists:
	\begin{itemize}
	\item [$e = read(v)$.] Since $e$ is not critical in $E$, $e$ is not the first remote read of $v$ by $p$ in $E$, and since $F \mid p = E \mid p$, $e$ is not the first remote read of $v$ by $p$ in $F$, thus $e$ is a not critical in $F$.
	\item [$e = write(v)$.] Since $e$ is not critical in $E$, $p$ is the last process to commit a write to $v$ before $e$ in $E$. Denote this write event by $e'$. Since $F \mid p = E \mid p$, $e'$ appears in $F$ as well. There is no write commit between $e'$ and $e$ in $E$, and since $F \preceq E$ there is no write commit between $e'$ and $e$ in $F$ as well. Therefore $p$ is the last process to commit a write to $v$ before $e$ in $F$, thus $e$ is not critical in $F$.
	\end{itemize}
	
	$\bullet$ Assume $e$ is a critical event in $E$. Then $e$ is either a critical read or write event in $E$. We consider each of the cases:
	\begin{itemize}
	\item [$e = read(v)$.] Then $e$ is the first remote read of $v$ by $p$ in $E$. Since $F \mid p = E \mid p$, $e$ is the first remote read of $v$ by $p$ in $F$, thus a critical read in $F$.
	\item [$e = write(v)$.] Thus, the last process to commit a write to $v$ before $e$ in $E$ (if any) is not $p$. Since $F \mid p = E \mid p$, no write commit by $p$ has been added to $F$, and since $E \preceq F$, no write commit by another process has been removed, thus the last process to commit a write to $v$ before $e$ in $F$ (if any) is not $p$, and $e$ is a critical write in $F$.
	\end{itemize}
\end{proof}





\begin{lemma-repeat} {lem: sub-execution}
	Let $E$ be an execution and let $p \in P$ be a process such that $p \notin AW(q,E)$ for any $q \neq p$. Then $E^{-p}$ is an execution.
\end{lemma-repeat}

\begin{proof}
	We prove the claim by induction on the number of events in $E^{-p}$. The base case $E^{-p} = \langle \rangle$ is trivial. For the induction, let $E^{-p} = F f$ such that $E = E_1 f E_2$, $F = E_1^{-p}$, and $E_2$ is a (possibly empty) solo execution by $p$. Assume that $F$ is an execution, and let $q$ be the process that executes $f$.
	
	Process $q$ executes the same events in $E_1$ and in $F$, thus it is in the same state after both executions and about to execute the same event $f$. If $f$ is a transition or fence event, then $Ff$ is clearly an execution. This is the case also if $f$ is a write event. Otherwise, assume $f = read(v)$. The following two cases exist.
\begin{itemize}
\item Event $f$ reads a copy of $v$ from $q$'s write buffer in execution $E$. In this case, since $q$ is in the same state after $F$ and $E_1$, it will read the same value from its write buffer after $F$.
\item Otherwise, $f$ reads $v$ from the shared memory. Since $p \notin AW(q,E)$, the last process that commits a write to $v$ before $f$ in $E_1$ is not $p$. Hence, since all the processes except $p$ execute the same events in $E_1$ and in $F$, $v$ has the same writer and value after both executions. Consequently, $f$ reads the same value in $Ff$ and in $E_1 f$, so $Ff$ is an execution.
\end{itemize}
\end{proof}





\begin{lemma-repeat} {lemma: erase-invis-procs}
	Let $E$ be an execution, $\mathit{INV}$ be an IN-set of $E$ and $Y \subseteq \mathit{INV}$.
	\newline Define $F = E^{-Y}$. Then the following hold:
	\begin{enumerate}
		\item $F$ is an execution;
		\item $Act(F) = Act(E) \setminus Y$ and $Fin(F) = Fin(E)$;
		\item $\mathit{INV} \setminus Y$ is an IN-set of $F$;
		\item Each $p \in Act(F)$ executes the same critical events in $F$ and in $E$.
	\end{enumerate}
\end{lemma-repeat}


\begin{proof}  \mbox{}
	
	We first prove the claim for the case $Y = \{p\}$, a single process.
\begin{enumerate}
	
\item
	Consider $q \in P$ different from $p$. Since $p \in INV$, by IN1 $q$ is not aware of $p$ in $E$, i.e. $p \notin AW(q,E)$. By Lemma \ref{lem: sub-execution} $E^{-p}$ is an execution.
	
\item
	We removed the events of $p \in Act(E)$, thus $Act(E') = Act(E) \setminus \{p\}$ and $Fin(E') = Fin(E)$.
	
\item
	We prove $INV \setminus \{p\}$ is an IN-set of $E'$:
	
	IN1: Consider $q \neq p$. Since $E \mid q = E' \mid q$, we have $AW(q,E) = AW(q,E')$. By IN1, $AW(q,E) \cap INV \subseteq \{q\}$, in particular $AW(q,E') \cap INV \setminus \{p\} \subseteq \{q\}$. $p$ executes no events in $E'$, thus $AW(p,E') = \emptyset$ and IN1 holds for $p$ in $E'$.
	
	IN2: For any $q \in INV \setminus \{p\}$, we have $E' \mid q = E \mid q$, thus $status(q,E') = status(q,E) = entry$.
	
	IN3: Consider $Z \subseteq INV \setminus \{p\}$, and $e \in E^{'-Z} = E^{-Z \cup \{p\}}$. Notice that $e \in E, E^{-p}, E^{'-Z}$.  Since $Z, \{p\} \subseteq INV$ and $INV$ is an IN-set of $E$, by IN3 applied to $E$ we have: $e$ is a critical event in $E^{-Z \cup \{p\}} = E^{'-Z}$ if and only if $e$ is a critical event in $E$ if and only if $e$ is a critical event in $E' = E^{-p}$.
	
	IN4: Consider an event $e \in E'$ by process $q$ accessing a remote variable $v$. Since $e \in E$ accesses a remote variable $v$, by IN4 $owner(v) \notin Act(E)$, and thus $owner(v) \notin Act(E') \subseteq Act(E)$.
	
	IN5: Assume $|Accessed(v,E') \cap Act(E')| > 1$ for some $v$. Since $E' \preceq E$ and $Act(E') \subseteq Act(E)$ we get $|Accessed(v,E) \cap Act(E)| > 1$, and by IN5 applied to $E$, $writer(v,E) \notin INV$. The only events removed are by $p \in INV$, thus $writer(v,E') = writer(v,E) \notin INV$, and in particular $writer(v,E') \notin INV \setminus \{p\}$.
	
\item
	Follows directly from IN4 for $E$ and the fact that $p \in INV$.
	
\end{enumerate}	

	For the general case we prove the claim by induction on $|Y|$. The base case $|Y| = 1$ has just been proven. Assume we proved the claim for any $|Y|=n$, and consider $Y \subseteq INV$ such that $|Y| = n+1$. Fix an arbitrary $p \in Y$, and denote $Z = Y \setminus \{p\}$. Then $Z \subseteq INV$ and $|Z|=n$. Denote $F_Z = E^{-Z}$, then by induction hypothesis:
	\begin{enumerate}
		\item $F_Z$ is an execution;
		\item $Act(F_Z) = Act(E) \setminus Z$, and $Fin(F_Z) = Fin(E)$;
		\item $INV \setminus Z$ is an IN-set of $F_Z$;
		\item Each $q \in Act(F_Z)$ executed the same critical events in $F_Z$ and in $E$;
	\end{enumerate}
	Notice that $F = F_Z^{-p}$ and $p \in INV \setminus Z$, thus using the induction base with $F_Z$ and $\{p\}$ we get:
	\begin{enumerate}
		\item $F$ is an execution;
		\item $Act(F) = Act(F_Z) \setminus \{p\} = Act(E) \setminus Y$, and $Fin(F) = Fin(F_Z) = Fin(E)$;
		\item $(INV \setminus Z) \setminus \{p\} = INV \setminus Y$ is an IN-set of $F$;
		\item Each $q \in Act(F)$ executed the same critical events in $F$ and in $F_Z$, and thus the same critical events in $F$ and in $E$.
	\end{enumerate}
\end{proof}

