\section{Full Lower Bound Proofs}
\label{sec:FullProof}

We start by stating a few lemmas and claims that are required for arguing about the properties of our construction, which we specify in a formal manner later. The proofs are technical and appear in the appendix.

\begin{claim-num}  \label{claim: maintain-criticalness}
	Let $E$ be an execution fragment and $e \in E$ be an event issued by some process $p$.
	\begin{itemize}
		\item Assume $e$ is a non-special event in $E$. Then for any execution fragment $F \preceq E$ such that $F \mid p = E \mid p$, $e$ is a non-special event in $F$.
		\item Assume $e$ is a special event in $E$. Then for any execution fragment $F$ such that $E \preceq F$ and $F \mid p = E \mid p$, $e$ is a special event in $F$.
	\end{itemize}
\end{claim-num}

\begin{lemma} \label{lem: sub-execution}
	Let $E$ be an execution and let $p \in P$ be a process such that $p \notin AW(q,E)$ for any $q \neq p$. Then $E^{-p}$ is an execution.
\end{lemma}

\begin{lemma}  \label{lem: access-visible-variable}
	Let $E$ be an execution and let $\mathit{INV}$ be an IN-set of $E$. Let $e$ be a $read(v)$ or $write(v)$ event. Assume $writer(v,E) \notin \mathit{INV}$ and $owner(v) \notin Act(E)$. Then $\mathit{INV}$ satisfies IN1-IN4 of Definition \ref{def:inv} in $E e$.
\end{lemma}

\begin{claim} \label{claim: local-event-extension}
	Let $E$ be an execution and let $\mathit{INV}$ be an IN-set of $E$. Let $e$ be an extension of $E$ by some process $p$ such that $e$ is a local event in $E e$. Then $\mathit{INV}$ is an IN-set of $E e$.
\end{claim}

\begin{lemma} \label{lem: non-critical-extension}
	Let $E$ be an execution and let $\mathit{INV}$ be an IN-set of $E$. Let $F$ be an extension of $E$ such that $F$ contains no critical or transition event in $E F$. Then $\mathit{INV}$ is an IN-set of $E F$.
\end{lemma}

\begin{lemma} \label{lem: remove-invisibale-processes}
	Let $E$ be an execution, $\mathit{INV}$ be an IN-set of $E$ and $Y \subseteq \mathit{INV}$.
	\newline Define $E' = E^{-Y}$. Then the following hold:
	\begin{enumerate}
		\item $E'$ is an execution;
		\item $Act(E') = Act(E) \setminus Y$ and $Fin(E') = Fin(E)$;
		\item $\mathit{INV} \setminus Y$ is an IN-set of $E'$;
		\item Each $p \in Act(E')$ executes the same critical events in $E'$ and in $E$;
		\item If $p \in Act(E')$ is about to execute a special event $f_p$ after $E$, then $p$ is about to execute a special event $e_p \sim f_p$ after $E'$.
	\end{enumerate}
\end{lemma}

\begin{lemma} \label{lem: next-critical-extension}
	Let $E$ be a regular execution. Then there exists an extension $F$ such that the following hold:
	\begin{itemize}
		\item $F$ contains no special events in $E F$;
		\item $E F$ is a regular execution;
		\item Each $p \in Act(E)$ is about to execute a special event $f_p$ after $E F$. Moreover, at most one process $p \in Act(E)$ is about to execute $f_p=CS_p$ after $E F$.
	\end{itemize}
\end{lemma}

The proof of the following theorem appears in \cite{Bollobas2004}.

\begin{theorem} [Tur\'{a}n] ~\label{th: turan}
	Let $\mathcal{G} = (V,E)$ be an undirected graph, with vertex set $V$ and edge set $E$. If the average degree of $\mathcal{G}$ is $d$, then an independent set exists with at least $\lceil |V|/(d+1) \rceil$ vertices.
\end{theorem}
	
We now prove a tradeoff between the fence complexity and the adaptivity function $f$.
We start with the regular execution $H_0$ in which each process $p$ have executed the $Enter_p$ event only, hence $Act(H_0)=P$ and $Fin(H_0)=\emptyset$. We then build longer executions $H_1,H_2,...$ inductively. At each induction step, we construct $H_{i+1}$ from $H_i$ using three phases: read, write, and regularization. Each phase consists of a sequence of executions.

Every induction step starts with an execution $H_i$ that meets the following conditions:

\begin{enumerate}[(a)]
	\item $H_i$ is a regular execution;
	\item Each $p \in Act(H_i)$ executes $\ell_i$ critical events in $H_i$, for some $\ell_i \leq f(i)$;
	\item $|Fin(H_i)| = i$;
	\item Each $p \in Act(H_i)$ completes $i$ fences in $H_i$ and $mode(p,H_i) = read$.
\end{enumerate}

For simplicity, we slightly abuse notation and write $\ell$ instead of $\ell_i$ in the rest of this section.




\subsection{Read phase}
\label{subsec:read-phase}
In the course of the read phase, we construct a sequence of executions $H_i=G_0,G_1,G_2,\ldots,G_s, J_0$.

\begin{lemma} \label{read-phase-lemma}
	At each step during the read phase we have an execution $G_k$ satisfying the following conditions:
	\begin{enumerate}[(1)]
		\item $G_k$ is a regular execution;
		\item Each $p \in Act(G_k)$ executes $\ell+k$ critical events in $G_k$;
		\item Each $p \in Act(G_k)$ completes $i$ fences in $G_k$ and $mode(p,G_k) = read$;
		\item $Fin(G_k) = Fin(H_i)$;
		\item $|Act(G_k)| \geq (|Act(G_{k-1})|-1)/10$.
	\end{enumerate}
\end{lemma}

\vspace{3mm} First notice that $G_0 = H_i$ satisfies all the conditions in Lemma \ref{read-phase-lemma}. Assume we already constructed $G_{k-1}$ satisfying the conditions in Lemma \ref{read-phase-lemma}. We let $G=G_{k-1}$ and $n=|Act(G_{k-1})|$ in the rest of this section (\ref{subsec:read-phase}), in which we specify the construction of the read phase and prove Lemma \ref{read-phase-lemma}.

\subsubsection{Construction: stage 1}
By Lemma \ref{lem: next-critical-extension}, there exists an extension $F$ of $G$ such that the following hold:
\begin{enumerate}
	\item $F$ contains no special events in $G F$;
	\item $G F$ is a regular execution;
	\item $F$ contains no transition events, therefore $Act(G F) = Act(G)$ and $Fin(G F) = Fin(G) = Fin(H_i)$;
	\item Each $p \in Act(G)$ executes $\ell+k-1$ critical events in $G F$;
	\item $F$ contains no fence events, hence each $p \in Act(G)$ completes $i$ fences in $G F$ and $mode(p,G F) = mode(p,G) = read$;
	\item Each $p \in Act(G)$ is about to execute a special event $f_p$ after $G F$. Moreover, at most a single process $q \in Act(G)$ is about to execute $f_q=CS_q$.
\end{enumerate}
Denote by $Y$ the set of processes in $Act(G)$ such that $f_p \neq CS_p$. We have $n-1 \leq |Y| \leq n$. For each $p \in Y$, since $status(p,G F) = entry$ and $f_p \neq CS_p$ we get that $f_p$ is not a transition event, and since $mode(p,G F) = read$ we get that $f_p$ is either a read event or a $BeginFence$ event.
\\ We define: $Z_1 = \{p \in Y \mid f_p = BeginFence\}$, $Z_2 = \{p \in Y \mid f_p \text{ is a read event}\}$.
\\ It follows that $Y = Z_1 \cup Z_2$ and $Z_1 \cap Z_2 = \emptyset$, thus $|Y|=|Z_1|+|Z_2|$.

\subsubsection*{\hspace{5mm} Case I: $\bf |Z_1| > |Y|/2$}
We define $W = Z_1$. We have $|W| > |Y|/2 \geq (n-1)/2$, thus $|W| \geq n/2$.

\subsubsection*{\hspace{5mm} Case\ II: $\bf |Z_2| \geq |Y|/2$}
We construct an undirected graph $\mathcal{G}$ as follows: the vertices of $\mathcal{G}$ are the processes in $Z_2$. Consider $p \in Z_2$ and denote $f_p = read(v)$. We add an edge $\{p,q\}$ if there exists $q \in Z_2$ such that $v$ is local to $q$ or $writer(v, G F) = q$.
\\ Since $v$ is local to at most one process and has at most one last writer, $p$ accounts for at most 2 edges in $\mathcal{G}$, thus the average degree in $\mathcal{G}$ is at most 4. By Theorem \ref{th: turan}, there exists an independent set $W \subseteq Z_2$ in $\mathcal{G}$ such that:
$$|W| \geq |Z_2|/5 \geq |Y|/10 \geq (n-1)/10$$

\subsubsection{Construction: stage 2}
We have a set of processes $W \subseteq Act(G F)$. Define $\overline{W} = Act(G F) \setminus W$. By Lemma \ref{lem: remove-invisibale-processes} with $'E' \leftarrow G F$ and $'Y' \leftarrow \overline{W}$, we have an execution $N = (G F)^{-\overline{W}}$ such that the following hold:
\begin{enumerate}
	\item $W = Act(G F) \setminus \overline{W}$ is an IN-set of $N$;
	\item $Act(N) = Act(G F) \setminus \overline{W} = W$, and $Fin(N) = Fin(G F) = Fin(H_i)$;
	\item From 1 and 2: $N$ is a regular execution;
	\item Each $p \in W$ executed the same critical events in $N$ and in $G F$, thus $p$ executed $\ell+k-1$ critical events in $N$;
	\item For each $p \in W$, since $N \mid p = (G F) \mid p$ we get that $p$ completed $i$ fences in $N$ and $mode(p,N) = mode(p, G F) = read$;
	\item Each $p \in W$ is about to execute a special event $e_p \sim f_p$ after $N$.
\end{enumerate}

We extend $N$ by letting each $p \in W$ execute its next event in an arbitrary order. Denote this extension by $D$, and define $G_k = N D$. Notice that $Act(G_k) = Act(N) = W$ and $Fin(G_k) = Fin(N) = Fin(H_i)$ since $D$ contains no transition events.

We now analyze the resulting execution, $G_k$, according to the cases defined in stage 1.
\subsubsection*{\hspace{5mm} Case I}
Define $s=k-1$ and $J_0 = G_k$. For each $p \in W$ we have $e_p \sim f_p = BeginFence$. The following conditions hold.
\begin{enumerate}
	\item $D$ contains fence events only, thus by Lemma \ref{lem: non-critical-extension}, $W$ is an IN-set of $J_0 = N D$, i.e. $J_0$ is a regular execution;
	\item Each $p \in Act(J_0)$ executed $\ell+s$ critical events in $N$ and thus in $J_0$;
	\item Each $p \in Act(J)$ completed $i$ fences in $J_0$, and the last event by $p$ in $J_0$ is BeginFence, i.e. $mode(p,J_0) = write$;
	\item $Fin(J_0) = Fin(H_i)$;
	\item $|Act(J_0)| = |W| \geq |Act(G_s)|/2$;
\end{enumerate}

We are done with the read phase, and we proceed to the write phase.

\subsubsection*{\hspace{5mm} Case II}
By the definition of $W$, each $p \in W$ executed a single read event $e_p$ in $D$.

\begin{claim-subsection} \label{claim: G_k-is-regular}
	$G_k$ is a regular execution.
\end{claim-subsection}

\begin{proof}	
	Consider $p \in W$, and denote $e_p = read(v)$.
	
	property 1: $writer(v,N) \notin W$.
	\\ Denote $q = writer(v, G F)$. If $q \notin Act (G F)$ then after removing the events by processes in $\overline{W} \subseteq Act(G F)$ we still have $writer(v,N) = writer(v,G F) = q$ and $q \notin W \subseteq Act(G F)$. Otherwise  $q \in Act(G F)$. $G F$ is a regular execution in which $q$ accessed $v$ and $writer(v, G F) \in Act(G F)$, thus by IN5 $q$ is the only process in $Act(G F)$ to access $v$. Notice that $q \notin W$ since there is an edge $\{p,q\}$ in $\mathcal{G}$, and $p \in W$, an independent set of $\mathcal{G}$. Therefore $q \in \overline{W}$, and after removing events by processes in $\overline{W}$ there is no process in $Act(G F)$ (and thus in $W$) to access $v$ in $N$, that is $writer(v,N) \notin W$.
	
	property 2: $owner(v) \notin W$:
	\\ Denote $q_v = owner(v)$. If $q_v \notin Z_2$ the claim clearly hold. Otherwise $q_v \in Z_2$ and $f_p \sim e_p = read(v)$, thus $\mathcal{G}$ contains an edge $\{p,q_v\}$ (notice that $p \neq q_v$ since $p$ remotely reads $v$). Since $p \in W$ and $W$ is an independent set we have $q_v \notin W$.
	
	Denote by $D_j$ the prefix of $D$ that contains exactly $j$ events. We prove by induction on $j$ ($0 \leq j \leq |D|$) that $W$ is an IN-set of $N D_j$:
	\\ induction base $j=0$: by our construction $W$ is an IN-set of $N$.
	\\ Assume we already proved the claim for $j < |D|$. Notice that $N D_{j+1} = N D_j e_p$ for some $p \in W$, and denote $e_p = read(v)$. Since $D_j$ contains no transition events, we have $Act(N D_j) = Act(N) = W$. $D_j$ contains only read events, thus no write to $v$ occurs in $D_j$, i.e. $writer(v, N D_j) = writer(v, N) \notin W$. Together with the fact that $owner(v) \notin W$, the conditions for Lemma \ref{lem: access-visible-variable} holds, and $W$ satisfies IN1-IN4 in $N D_j e_p = N D_{j+1}$. As IN5 holds for $W$ in $N D_j$ it clearly holds for any variable $u \neq v$ in $N D_{j+1}$. Since $e_p$ is a read event to $v$ we get $writer(v, N D_{j+1}) = writer(v, N D_j) \notin W$ and IN5 holds for $v$ in $N D_{j+1}$.
	
	Using the last claim with $j = |D|$ we have that $W$ is an IN-set of $N D = G_k$, and thus $G_k$ is a regular execution.
\end{proof}
	
We now prove that $G_k$ satisfies all the conditions in Lemma \ref{read-phase-lemma}:
\begin{enumerate}[(1)]
	\item $G_k$ is a regular execution;
	\item Consider $p \in W = Act(G_k)$. $p$ executed $\ell+k-1$ critical events in $N$ and a single event $e_p$ in $D$. By our construction $e_p$ is a critical event in $N e_p$, $N e_p \preceq G_k$, and $(N e_p) \mid p = G_k \mid p$. Therefore, by claim \ref{claim: maintain-criticalness}, $e_p$ is a critical event in $G_k$. Altogether $p$ executed $\ell+k$ critical events in $G_k$;
	\item Consider $p \in W$. $p$ completed $i$ fences in $N$ and $mode(p,N) = read$. Since $D$ contains a single read event by $p$ we get that $p$ completed $i$ fences in $G_k$ and $mode(p,G_k) = read$;
	\item $Fin(G_k) = Fin(N) = Fin(H_i)$;
	\item $|Act(G_k)| = |W| \geq (n-1)/10$.
\end{enumerate}

\begin{claim-section} \label{claim: read-upper-bound}
	The number of steps in the read phase is bounded by $f(i+1)-\ell$, that is, $\ell+s \leq f(i+1)$.
\end{claim-section}

\begin{proof}
	Assume towards a contradiction that during the read phase we build an execution $G_k$ such that $\ell+k > f(i+1)$. Then $G_k$ satisfies:
	\begin{itemize}
		\item $G_k$ is a regular execution;
		\item Each $p \in Act(G_k)$ executed $\ell+k$ critical events in $G_k$;
		\item $Fin(G_k) = Fin(H_i)$, thus $|Fin(G_k)| = i$.
	\end{itemize}
	We choose an arbitrary $p \in Act(G_k)$ and denote $Y = Act(G_k) \setminus \{p\}$. Using Lemma \ref{lem: remove-invisibale-processes} with $'E' \leftarrow G_k$ and $'Y' \leftarrow Y$, we have an execution $G'_k = G_k^{-Y}$ such that: $Act(G'_k) = \{p\}$ and $Fin(G'_k) = Fin(G_k) = Fin(H_i)$; $p$ executes the same critical events in $G_k$ and in $G'_k$, thus $p$ executes $\ell+k$ critical events in $G'_k$. Hence, at most $i+1$ processes issue events in $G'_k$, i.e. the total contention of $G'_k$ is at most $i+1$. However, $p$ executes $\ell+k > f(i+1)$ critical events during a single passage in $G'_k$, a contradiction to our assumption that the algorithm is $f$-adaptive.
\end{proof}




\subsection{Write phase}
At the end of the read phase we get an execution $J$ such that the last event by each process in $Act(J)$ was BeginFence. The write phase will determine the order in which processes are executing their write commits along the fence. Each process $p \in Act(J)$ has a list $\alpha_p$ of writes in its write buffer, which it will commit starting from $J$.

We first focus on the list $\alpha_p$. Notice that $\alpha_p$ is also a solo run of $p$ starting from $J$ until the point where it finish executing its fence. Along the write phase, we first construct a set of processes out of $Act(J)$, such that the lists $\alpha_p$ of those processes satisfies some special properties. Then, we will use this set in order to extend $J$.
In the course of the write phase we construct a sequence of sets $Act(J) = P_0 \supseteq P_1 \supseteq \ldots \supseteq P_t$ satisfying the Lemma below.

\begin{lemma} \label{lemma-new}
	At each step of the write phase we have a set $P_k \subseteq Act(J)$ such that each process $p \in P_k$ has a prefix $\beta_p$ of $\alpha_p$ and the following hold:
	\begin{enumerate}[(1)]
		\item For each $p \in P_k$, $p$ is executing exactly $k$ critical writes in $\beta_p$ if it to perform it after $J$.
		\item Let $e \in \beta_p$ be a critical write in $J \beta_p$ by some process $p \in P_k$ to some variable $v$. Then, the following holds:
		\begin{enumerate}
			\item $writer(v,J) \notin P_k$.
			\item $owner(v) \notin P_k$.
			\item Either $p$ is the only process in $P_k$ to write to $v$ in its $\beta_p$, or that any process $q \in P_k$ has a write to $v$ in $\beta_q$.
		\end{enumerate}
		\item $|P_k| \geq \sqrt{|P_{k-1}|}/10$.
	\end{enumerate}
\end{lemma}

First notice that $P_0=Act(J)$ satisfies all the above conditions with $\beta_p=\langle \rangle$ for any $p \in Act(J)$. Assume we already constructed $P_{k-1}$ satisfying the conditions of Lemma \ref{lemma-new}. We denote $n=|P_{k-1}$, and for any $p \in P_{k-1}$ let $\beta_p$ to be the prefix as promised by Lemma \ref{lemma-new}, throughout the rest of this section, in which we define the construction of the write phase, and prove Lemma \ref{lemma-new}.

\subsubsection{Construction: stage 1}
For any $p \in P_{k-1}$ we let $p$ perform a solo run starting from $J \beta_p$ until its next event $e_p$ is either a critical write, or that $p$ has finished executing $\alpha_p$ and its next event $e_p$ is $EndFence$. Denote this extension by $\gamma_p$. Notice that $\beta_p \gamma_p$ is indeed a prefix of $\alpha_p$ by its definition.

Let $Z$ be the set of all processes that have not finished executing $\alpha_p$, that is, $Z = \{p \in P_{k-1} \mid \beta_p \gamma_p \neq \alpha_p\}$.
If $|Z|<|P_{k-1}|/2$, i.e., at least half of the processes in $P_{k-1}$ have finished executing $\alpha_p$, then we define $t=k-1$ and $P' = P_{k-1} \setminus Z$, and we move to stage 3 of the write phase.
Otherwise $|Z| \geq |P_{k-1}|/2$. We define $V_{next}$ to be all the variables that are about to be written in one of the next events $e_p$ by the processes in $Z$. Formally, $V_{next} = \{v \in V \mid \exists p \in Z \text{ such that } e_p \text{ remotely writes v}\}$. In order to construct $P_k$ the following stage will handle the cases of low and high contention separately.

\subsubsection{Construction: stage 2}
\subsubsection*{\hspace{5mm} Case I: $\bf |V_{next}| < \sqrt{|Z|}$}
By the pigeonhole principle, there exists a variable $u \in V_{next}$ and a set $P_k \subseteq Z$ of size $|P_k| \geq \sqrt{|Z|} \geq \sqrt{n/2}$, such that $e_p$ is a critical commit write to $u$ for any $p \in P_k$. We now prove that $P_k$ satisfies all the conditions of Lemma \ref{lemma-new}, where for each $p \in P_k$ we choose $\beta_p' = \beta_p \gamma_p e_p$ to be the prefix of $\alpha_p$. We will first prove condition (2) holds.

Consider some $p \in P_k$. We first prove that $p$ is not allowed to write any variable in $\gamma_p e_p$ that was written by some other process $q \in P_k$ in $\beta_q$. Assume towards a contradiction that there is such $q$.

it is not allowed to write any variable $v$ that was written to by some process

Let $e \in \beta_p'$ be a critical write in $J \beta_p'$ by some $p \in P_k$ to some variable $v$. First assume $e \in \beta_p$. Since $P_k \subseteq P_{k-1}$ conditions (a) and (b) follows immediately. In addition, if there is a process $q \in P_k$ which writes to $v$ in $\beta_q'$, then the first such write in $J \beta_q'$ is critical (by conditions (a) and (b)). If this write is in $\beta_q$, then by induction, all processes in $P_k \subseteq P_{k-1}$ has a write to $v$ in $\beta_q$, and thus in $\beta_q'$. Otherwise, this write must be $e_q$, since $\gamma_q$ contains no critical writes. It follows that $v=u$, contradicting the fact that $e_p$, the last event in $J \beta_p'$ is critical, as $\beta_p'$ is a solo run by $p$ that contains a write to $u=v$ before $e_p$.


any process $q \in P_{k-1}$ has a write to $v$ in $\beta_q$, then so is the case for $q \in P_k$ with $\beta_q'$. Otherwise, if there is $q \in P_k$ that writes to $v$ in $\beta_q'$, as 

no process $q \in P_k$ different then $p$ writes to $v$ in $\beta_q$. If there is sch $q$ that writes to $v$ in
consider some $q \in P_k$. If there is a write to $v$ in $\beta_q'$, then the first such write in $J \beta_q'$ is critical (as $writer(v,J) \notin P_k$). 

\begin{enumerate}[(1)]
	\item For each $p \in P_k$, by induction hypothesis $p$ executes $k-1$ critical writes along $\beta_p$ in the run $J \beta_p$. We extended $\beta_p$ with a solo run such that $\gamma_p$ contains no critical events, end $e_p$ is a critical event. Therefore $p$ executes $k$ critical events along $\beta_p' = \beta_p \gamma_p e_p$ in the run $J \beta_p'$.
	\item Let $e \in \beta_p'$ be a critical write in $J \beta_p'$ by some $p \in P_k$ to some variable $v$. First assume $e \in \beta_p$. Then, since $P_k \subseteq P_{k-1}$ conditions (a) and (b) follows immediately. In addition, if all processes in $P_{k-1}$ 

\end{enumerate}










\begin{align*}
Z_1 &= \{p \in P_{k-1} \mid e_p = EndFence\} \\
Z_2 &= \{p \in P_{k-1} \mid e_p \text{ is a commit write event}\} \\
V_{next} &= \{v \in V \mid \exists p \in Z_2 \text{ such that } e_p \text{ remotely writes v}\}
\end{align*}
It follows that $ = Z_1 \cup Z_2$ and $Z_1 \cap Z_2 = \emptyset$, thus $n = |Z_1|+|Z_2|$.



In the course of the write phase we construct a sequence of executions $J_0,J_1,\ldots,J_t, L_0$. Denote $\delta = \ell+s$.

\begin{lemma} \label{write-phase-lemma}
	At each step during the write phase we have an execution $J_k$ such that $J_k$ can be written as $\alpha \beta$, and the following conditions holds:
	\begin{enumerate}[(1)]
		\item $\alpha$ is a regular execution in which every $p \in Act(J_k)$ executed $\delta$ critical events.
		\item $\beta$ is an extension in which all processes are executing a (single) fence, such that each $p \in Act(J_k)$ committed $k$ critical writes in $\beta$.
		\item For any $v \in V$ such that there exists a critical write to $v$ in $\beta$ by some process $p$ the following holds:
		\begin{enumerate}
			\item $writer(v,\alpha) \notin Act(J_k)$.
			\item $owner(v) \notin Act(J_k)$.
			\item Either $p$ is the only process in $Act(J_k)$ to commit a write to $v$ in $\beta$, or that any process in $Act(J_k)$ committed a write to $v$ in $\beta$.
		\end{enumerate}
		\item Each $p\in Act(J_k)$ completed $i$ fences in $J_k$, and $mode(p, J_k) = write$;
		\item $Fin(J_k) = Fin(H_i)$;
		\item $|Act(J_k)| \geq \sqrt{|Act(J_{k-1})|}/4(\delta+k)$.
	\end{enumerate}
\end{lemma}

First notice that $J_0$ satisfies all the conditions in Lemma \ref{write-phase-lemma} by choosing $\alpha = J_0$ and $\beta = \langle \rangle$, since $J_0$ is a regular execution. Assume we already constructed $J_{k-1}$ satisfying the conditions of Lemma \ref{write-phase-lemma}. We denote $J=J_{k-1},\ n=|Act(J_{k-1})|$, and $J = \alpha \beta$ the composition of $J$ by Lemma \ref{write-phase-lemma}, throughout the rest of this section in which we define the construction of the write phase, and prove Lemma \ref{write-phase-lemma}.

\subsubsection{Construction: stage 1}
For $p \in Act(J)$, since $p$ is executing a fence after $J$ it can continue and execute commit write events independently of other processes. We extend $J$ in the following manner: we let each $p \in Act(J)$ (in an arbitrary order) run until it reaches its next special event $e_p$. Such an event must exist, since $p$ is executing a fence. Denote this extension by $\gamma$. Then $J \gamma$ is an execution and the following hold:
\begin{enumerate}
	\item $\gamma$ contains no transition events, thus $Act(J \gamma) = Act(J)$ and $Fin(J \gamma) = Fin(J) = Fin(H_i)$.
	\item Each $p \in Act(J)$ executed $\delta+k-1$ critical events in $J \gamma$.
	\item $\gamma$ contains no fence events, thus each $p \in Act(J \gamma)$ completed $i$ fences in $J \gamma$.
	\item For any $p \in Act(J)$, since $\gamma$ contains no fence events, $mode(p,J \gamma) = mode(p,J) = write$. Therefore $e_p$ is either a commit write event or an $EndFence$ event.
	
	%Moreover, as $e_p$ is a special event in $J F' e_p$, for some prefix $F'$ of $F$, and since we extend $F'$ with events by processes other than $p$ to form $F$, by claim \ref{claim: maintain-criticalness} we have that $e_p$ is a special event in $J F e_p$, i.e. $p$ is about to execute a special event $e_p$ after $J F$.
\end{enumerate}

We define:
\begin{align*}
Z_1 &= \{p \in Act(J) \mid e_p = EndFence\} \\
Z_2 &= \{p \in Act(J) \mid e_p \text{ is a commit write event}\} \\
V_{next} &= \{v \in V \mid \exists p \in Z_2 \text{ such that } e_p \text{ remotely writes v}\}
\end{align*}
It follows that $Act(J) = Z_1 \cup Z_2$ and $Z_1 \cap Z_2 = \emptyset$, thus $n = |Z_1|+|Z_2|$.

\subsubsection*{\hspace{5mm} Case I: $\bf |Z_1| \geq n/2$}
We define $W = Z_1$.

\subsubsection*{\hspace{5mm} Case II: $\bf |Z_2| > n/2$ and $\bf |V_{next}| \geq \sqrt{|Z_2|}$}
For each $v \in V_{next}$, we select an arbitrary $p \in Z_2$ such that $e_p = write(v)$. Denote the set of these processes by $Z$. Then, $|Z|=|V_{next}| \geq \sqrt{|Z_2|}$.
\newline We construct an undirected graph $\mathcal{G}$ as follows: the vertices of $\mathcal{G}$ are the processes in $Z$. Consider $p \in Z$, and denote $e_p = write(v)$. For $q \in Z$, we add an edge $\{p,q\}$ in $\mathcal{G}$ if either
\begin{inparaenum}[\itshape a\upshape)]
	\item $v$ is local to $q$; or
	\item $writer(v,\alpha) = q$.
	%there exists a critical event $e \in J F$ by $q$ such that $e$ remotely accesses $v$.
\end{inparaenum}
\\ Since each variable $v$ is local to at most one process, and has at most a single process as $writer(v,\alpha)$, at most two new edges are introduced by each process, and therefore the average degree in $\mathcal{G}$ is at most 4. By Theorem \ref{th: turan}, there exists an independent set $W \subseteq Z$ in $\mathcal{G}$ such that:

$$ |W| \geq \frac{|Z|}{5} \geq \frac{\sqrt{|Z_2|}}{5} \geq \frac{\sqrt{n/2}}{5} \geq \frac{\sqrt{n}}{10} $$

\subsubsection*{\hspace{5mm} Case III: $\bf |Z_2| > |Y|/2$ and $\bf |V_{next}| < \sqrt{|Z_2|}$}
By the pigeonhole principle, there exists a variable $v$ and a set $W \subseteq Z_2$ of size $|W| \geq \sqrt{|Z_2|} \geq \sqrt{n/2}$, such that $e_p$ is a critical commit write to $v$ for any $p \in W$.

\subsubsection{Construction: stage 2}
Stage 1 defines a set $W \subseteq Act(J \gamma)$. Denote $\overline{W} = Act(J \gamma) \setminus W$, and $N = (J \gamma)^{-\overline{W}} = (\alpha \beta \gamma)^{-\overline{W}}$. Then the following hold:
\begin{enumerate}
	\item $Act(N) = Act(J \gamma) \setminus \overline{W} = W$ and $Fin(N) = Fin(J \gamma) = Fin(H_i)$.
	\item $\alpha$ is a regular execution, and $\overline{W} \subseteq Act(\alpha)$. By Lemma \ref{lem: remove-invisibale-processes} we get that $\alpha^{\overline{w}}$ is a regular execution and each $p \in Act(N)$ executed $\delta$ critical events in $\alpha^{\overline{w}}$.
	\item $Act(N) = Act(J F) \setminus \overline{W} = W$ and $Fin(N) = Fin(J F) = Fin(H_i)$.
	\item By IN3, each $p \in Act(N)$ executed the same critical events in $N$ and in $J F$, thus $p$ executed $\delta+k-1$ critical events in $N$.
	\item For any $p\in Act(N)$, as $N \mid p = (J F) \mid p$, $p$ completed $i$ fences in $N$ and $mode(p,N) = write$.
	\item Each $p \in Act(N)$ is about to execute $e_p$ after $N$.
\end{enumerate}

We extend $N$ by letting each process $p \in W$ execute its next event $e_p$ in an increasing ID order. Denote this extension by $D$, and define $J_k = N D$. Notice that no event in $D$ is a transition event, thus $Act(J_k) = Act(N) = W$ and $Fin(J_k) = Fin(N) = Fin(H_i)$.

We now analyze the resulting execution, $J_k$, according to the cases defined in stage 1.
\subsubsection*{\hspace{5mm} Case I}
Define $t = k-1$ and $L_0= J_k$. We have an execution $L_0 = N D$ where in $D$ each process in $Act(L_0)$ executes a single $EndFence$ event. The following conditions hold:
\begin{enumerate}
	\item $N$ is a semi-regular execution. It is easy to verify that fence events do not violate any of IN1-IN4, thus $L_0$ is a semi-regular execution as well;
	\item $N$ is an ordered execution.
	\item $D$ contains no critical events in $N D$, thus each $p \in Act(L_0)$ executes $\delta+t$ critical events in $L_0$;
	\item Each $p \in Act(L_0)$ completes $i$ fences in $N$, and the only event by $p$ in $D$ is $EndFence$. Therefore $p$ completes $i+1$ fences in $L_0$ and $mode(p,L_0) = read$;
	\item $Fin(L_0) = Fin(H_i)$;
	\item $|Act(L_0)| = |W| \geq |Act(J_t)|/2$.
\end{enumerate}

We are done with the write phase, and we proceed to the regularization phase.

\subsubsection*{\hspace{5mm} Case II + III}
By our construction, $D$ is a sequence of commit write events by processes in $W$ in an increasing ID order.

\begin{claim-subsection} \label{claim: writer-v-not-in-W}
	Let $e_p \in D$ be a commit write event and denote $e_p = write(v)$. Then $writer(v,N) \notin W$.
\end{claim-subsection}

\begin{proof}
	Since $J F$ is an ordered execution, one of (a)-(c) holds for $v$ in $J F$.
	
	If (a) holds, then $writer(v,J F) \notin Act(J F)$. Thus after removing events by processes in $\overline{W}$ we still have $writer(v,N) = writer(v,J F) \notin Act(J F)$. In particular, $writer(v,N) \notin W$.
	
	If (b) holds, then $writer(v,J F) = q$ for some $q \in Act(J F)$, and this is the only process in $Act(J F)$ to access $v$. Notice that $p \neq q$, as $e_p$ is critical in $J F e_p$. Assume towards a contradiction that $q \in W$. If case III holds, then $e_q$ accesses the same variable as $e_p$, that is, $e_q$ is a commit write to $v$, contradicting the fact that $e_q$ is critical in $J F e_q$. If case II holds, then either $q = owner(v)$, or $q$ remotely accessed $v$ in $J F$, and the first such event is critical. In both cases, $\mathcal{G}$ contains an edge $\{p,q\}$, contradicting the fact that $W$ is an independent set. Hence we have $q \in \overline{W}$, and after removing the events by $q$ there is no access to $v$ by any process in $Act(J F)$. In particular, $writer(v,N) \notin W$.
	
	If (c) holds, then $p$ already committed a write to $v$ during the current fence it is executing in $J F$, and $p$ is about to commit a write to $v$ after $J F$, contradicting the fact that a process is allowed to commit a write to a variable at most once during a single fence execution. Thus, (c) does not hold for $v$.
\end{proof}

Consider $p \in Act(J_k)$. Claim \ref{claim: writer-v-not-in-W} implies that $e_p$ is critical in $N e_p$. As the only event $p$ executes in $D$ is $e_p$, we get $(N e_p) \mid p = J_k \mid p$. Clearly $N e_p \preceq J_k$, thus, by Claim \ref{claim: maintain-criticalness}, $e_p$ is critical in $J_k$, that is, $D$ consists of critical commit writes in $J_k = N D$ by the processes in $W$ in an increasing ID order.

\begin{claim-subsection} \label{J_k-semi-regular}
	$J_k$ is a semi-regular execution.
\end{claim-subsection}

\begin{proof}
	$N$ is a semi-regular execution, therefore IN1-IN4 hold for $W$ in $N$.
	
	IN1: $D$ consists of commit writes only, thus no process changes its awareness set during $D$, that is $AW(p,J_k) = AW(p,N)$ for any process $p$, and IN1 holds in $J_k$.
	
	IN2: $D$ contains no transition events, thus for any $p \in W$: $status(p,J_k) = status(p,N) = entry$.
	
	IN3: Consider $X \subseteq Act(J_k)$. IN3 holds in $N$, thus it is enough to consider an event $e_p \in D^{-X}$. If $e_p$ is an $EndFence$ event, then it is non-critical in both $J_k$ and $J_k^{-X}$. Otherwise, $e_p$ is a critical write in $J_k$ to some variable $v$ by process $p \in W$. Denote $D = D_1 e_p D_2$. By claim \ref{claim: writer-v-not-in-W}, $writer(v,N) \notin W$, thus after removing events by processes in $X$ we still have $writer(v,N^{-X}) = writer(v,N) \notin W$. As $e_p$ is the only event by $p$ in $D$, we get that there is no event by $p$ in $D_1$, and thus in $D_1^{-X}$. Hence, either $D_1^{-X}$ contains a commit write to $v$ by a process different from $p$, and thus $writer(v,(N D_1)^{-X}) \neq p$, or there is no commit write to $v$ in $D_1$, and thus $writer(v, (N D_1)^{-X}) = writer(v, N^{-X}) \neq p$. In both cases, we get that $e_p$ is a critical write in $J_k^{-X}$.
		
	IN4: Consider an event $e \in J_k$ by process $p$ accessing a remote variable $v$. If $e \in N$, then, by IN4, $owner(v) \notin Act(N) = Act(J_k)$ and we are done. Thus assume $e \in D$. We prove the claim for each case separately. Notice that $p \in W$, since $e \in D$.
	\\ Case II: If $owner(v) \in Act(J_k) = W$, then $\mathcal{G}$ contains an edge $\{p,owner(v)\}$, contradicting the fact that $W$ is an independent set in $\mathcal{G}$. Therefore $owner(v) \notin Act(J_k)$.
	\\ Case III: all the events in $D$ commit writes to the same variable $v$. Each $q \in Act(J_k) = W$ commits a critical write to $v$ in $D$, hence $q \neq owner(v)$, so $owner(v) \notin Act(J_k)$.
\end{proof}

\begin{claim-subsection} \label{J_k-ordered}
	$J_k$ is an ordered execution.
\end{claim-subsection}

\begin{proof}
	First, consider a variable $v$ such that $v$ is not accessed in $D$. Since $N$ is an ordered execution, one of the following holds for $v$:
	\begin{enumerate} [(a)]
		\item $writer(v,N) \notin Act(N)$, thus $writer(v,J_k) = writer(v,N) \notin Act(J_k)$.
		\item $writer(v,N) = p$ for some $p \in Act(N)$, and $p$ is the only process in $Act(N)$ to access $v$ in $N$. Since there is no access of $v$ in $D$, $writer(v,J_k) = writer(v,N) = p$, and $p$ is the only process in $Act(J_k)$ to access $v$ in $J_k$.
		\item There is a sequence $C$ in $N$, such that $C$ is a sequence of successive commit writes to $v$ by all the processes in $Act(N)$ in an increasing ID order. Therefore $C$ is such a sequence in $J_k$ as well.
		\\ Moreover, any process $p \in Act(N)$ is executing a fence in $N$ in which it committed its write in $C$. Since $p$ executes a single write event $e_p$ is $D$, $p$ is executing the same fence in $J_k$. That is, $p$ is executing in $J_k$ the fence in which it committed its write in $C$.
	\end{enumerate}
	Consider now a variable $v$ that is accessed in $D$. We prove the claim for each case separately.
	
	Case II: By the definition of $W$, each event in $D$ is a commit write to a different variable, thus there is a unique event $e_p \in D$ accessing $v$, namely $writer(v,J_k) = p \in Act(J_k)$. Assume there is another process $q \in Act(J_k)$ that accessed $v$ in $J_k$. Since $e_p$ is the only event in $D$ to access $v$, $q$ accessed $v$ in $N$, and thus in $J F$. Either $q = owner(v)$ or $q$ executed a critical event accessing $v$ in $J F$ (the first such event). In both cases, $\mathcal{G}$ contains an edge $\{p,q\}$, contradicting the fact that $W = Act(J_k)$ is an independent set. Therefore $p$ is the only process in $Act(J_k)$ to access $v$ in $J_k$.
	
	Case III: By the definition of $W$, $D$ is a sequence of successive commit write events by all the processes in $Act(J_k)$, in an increasing ID order, all of which to the same variable $v$. Moreover, each $p \in Act(J_k)$ is executing a fence in $J_k$ and the last event by $p$ is the commit write to $v$ in $D$, that is, it executed this event during the current fence it is executing.
\end{proof}

We now prove that $J_k, 1 \leq k \leq t,$ satisfies all the conditions of Lemma \ref{write-phase-lemma}:
\begin{enumerate}[(1)]
	\item By Claims \ref{J_k-semi-regular} and \ref{J_k-ordered}, $J_k$ is a semi-regular ordered execution;
	\item Each $p \in Act(J_k)$ executes $\delta+k-1$ critical events in $N$, and a single critical event $e_p$ in $D$, thus $p$ executes $\delta+k$ critical events in $J_k$.
	\item Consider $p \in Act(J_k)$. $p$ completes $i$ fences in $N$ and $mode(p,N) = write$. The only event by $p$ in $D$ is a commit write, thus $p$ completes $i$ fences in $J_k$ and $mode(p,J_k) = write$;
	\item $Fin(J_k) = Fin(H_i)$;
	\item $|Act(J_k)| = |W| \geq \sqrt{n}/4(\delta+k)$.
\end{enumerate}

\begin{claim-section} \label{claim: write-upper-bound}
	The number of steps in the read and write phases is bounded by $f(i+1)-\ell$. In other words: $\ell+s+t \leq f(i+1)$.
\end{claim-section}

\begin{proof}
	Assume towards a contradiction that during the write phase we build an execution $J_k$ such that $\ell+s+k = \delta+k > f(i+1)$. Then $J_k$ satisfies:
	\begin{itemize}
		\item $J_k$ is a semi-regular ordered execution;
		\item Each $p \in Act(J_k)$ executes $\delta+k$ critical events in $J_k$;
		\item $Fin(J_k) = Fin(H_i)$, thus $|Fin(J_k)| = i$;
	\end{itemize}
	We choose an arbitrary $p \in Act(J_k)$, and denote $W = Act(J_k) \setminus \{p\}$. Using Claim \ref{claim: pseudo-regular ordered execution}, we have an execution $J'_k = J_k^{-W}$. Notice that $Act(J'_k) = Act(J_k) \setminus W = \{p\}$ and $Fin(J'_k) = Fin(J_k) = Fin(H_i)$. By IN3 applied to $J_k$, we get that $p$ executes the same critical events in both $J'_k$ and in $J_k$, that is, $p$ executes $\delta+k$ critical events in $J'_k$.
Hence, at most $i+1$ processes issue events in $J'_k$, i.e. the total contention of $J'_k$ is at most $i+1$. However, $p$ executes $\delta+k > f(i+1)$ critical events during a single passage in $J'_k$, a contradiction.
\end{proof}




\subsection{Regularization phase}
We have a semi-regular execution $L_0$. Moreover, $L_0 = N D$ where $N$ is an ordered execution and $D$ is a sequence of $EndFence$ events by the processes in $Act(L_0)$. Let $p_{max}$ be the process with the largest ID in $Act(L_0)$, and define $W_0 = Act(L_0) \setminus \{p_{max}\}$.

\begin{claim-subsection} \label{claim: W_0-is-an-IN-set}
	$W_0$ is an IN-set of $L_0$.
\end{claim-subsection}

\begin{proof}
	As $L_0$ is a semi-regular execution, IN1-IN4 hold for $Act(L_0)$. Since $W_0 \subset Act(L_0)$, IN1-IN4 hold for $W_0$ in $L_0$. We now prove that IN5 holds in $L_0$.
	
	Consider a variable $v$. Since $D$ contains only fence events, $Act(N) = Act(L_0)$ and $writer(v,L_0) = writer(v,N)$. As $N$ is an ordered execution, one of the following holds for $v$:
	\begin{enumerate} [(a)]
		\item $writer(v,N) \notin Act(N)$, thus $writer(v,L_0) \notin W_0$.
		\item $writer(v,N) = p$ for some $p \in Act(N)$, and $p$ is the only process in $Act(N)$ to access $v$ in $N$. Since there is no access of $v$ in $D$, $p$ is the only process in $Act(L_0)$ to access $v$ in $L_0$, that is $|Accessed(v,L_0) \cap Act(L_0)| = 1$.
		\item There is a sequence $C$ in $N$, such that $C$ is a sequence of successive commit writes to $v$ by all the processes in $Act(N)$ in an increasing ID order. Therefore, $p_{max}$ is the last process to commit write to $v$ in $C$. For any $p \in Act(N)$, $p$ is executing a fence after $N$, in which it executed its write in $C$. A process commits at most a single write to any specific variable during a fence execution, therefore there is no write by $p$ to $v$ after $C$ in $N$. Thus, either there is a commit write to $v$ after $C$ by a process not in $Act(N)$, and thus $writer(v,N) \notin Act(N)$, or there is no commit write to $v$ after $C$, in which case $writer(v,N) = p_{max}$. In both cases, we have $writer(v,L_0) \notin W_0$.
	\end{enumerate}
\end{proof}

In the regularization phase, we construct a sequence of executions $L_0,L_1, \ldots, L_m, H_{i+1}$. Denote $\ell_{i+1} = \ell+s+t$.

\begin{lemma} \label{regularization-phase-lemma}
	In each step, we have an execution $L_k$ such that the following conditions hold:
	\begin{enumerate}[(1)]
		\item $Act(L_k)$ can be written as $W_k \cup \{p_{max}\}$ (where $p_{max} \notin W_k$);
		\item $W_k$ is an IN-set of $L_k$;
		\item $p_{max}$ executed $\ell_{i+1}+k$ critical events in $L_k$;
		\item Each $p \in W_k$ executes $\ell_{i+1}$ critical events in $L_k$;
		\item Each $p \in W_k$ completes $i+1$ fences in $L_k$ and $mode(p,L_k) = read$;
		\item $Fin(L_k) = Fin(H_i)$;
		\item $|Act(L_k)| \geq |Act(L_{k-1})|-1$.
	\end{enumerate}
\end{lemma}

First notice that $L_0$ satisfies all the conditions of Lemma \ref{regularization-phase-lemma}. Assume we already constructed $L_{k-1}$ satisfying the conditions of Lemma \ref{regularization-phase-lemma}. We denote $L=L_{k-1},\ n=|Act(L_{k-1})|$ throughout the rest of this section, in which we define the construction of the regularization phase and prove Lemma \ref{regularization-phase-lemma}.

Lemma \ref{lem: non-critical-extension} implies that an extension containing no critical or transition events does not effect the IN-set, that is the IN-set remains the same after the extension. It is easy to verify that a transition event by a process not in the IN-set does not affect it as well (no variable is accessed, and the only process that changes its state is not in the IN-set). We therefore conclude that an extension by processes not in the IN-set which contains no critical events does not change the IN-set, and the next corollary follows.
\begin{corollary} \label{claim: q-extension}
	Let $F$ be an extension of $L$ by $p_{max}$ such that $F$ contains no critical events in $L F$. Then $W_{k-1}$ is an IN-set of $L F$.
\end{corollary}

Let $F$ be a solo extension of $L$ by $p_{max}$, where $p_{max}$ executes until it either terminates (that is, executes $Exit_{p_{max}}$), or until it is about to issue a critical event $f$. First, we prove that such an extension exists.

Assume towards a contradiction that the solo run $F$ by $p_{max}$ after $L$ is infinite, where $p_{max}$ does not finish a passage in $F$, and $F$ contains no critical events in $L F$. Consider a finite prefix $F'$ of $F$. $p_{max}$ does not finish a passage in $F$, thus $Act(L F') = Act(L)$. $F'$ contains no critical events in $L F'$, thus, by Corollary \ref{claim: q-extension}, $W_{k-1}$ is an IN-set of $L F'$. Using Lemma \ref{lem: remove-invisibale-processes} with $'E' \leftarrow L F'$, $'INV' \leftarrow W_{k-1}$ and $'Y' \leftarrow W_{k-1}$, we get an execution $L' = (L F')^{-W_{k-1}}$ such that $Act(L') = Act(L F') \setminus W_{k-1} = \{p_{max}\}$. Notice that $L'$ can be written as $L^{-W_{k-1}} F'$, since $F'$ is a solo run by $p_{max} \notin W_{k-1}$. We have an execution $L'$ in which there is a solo run $F'$ by $p_{max}$, where $p_{max}$ is the only active process along $F'$, and $p_{max}$ does not finish a passage. Since this holds for any prefix of $F$, $F'$ can be as long as we wish, thus contradicting the global progress property.

\subsubsection*{\hspace{5mm} Case I}
$p_{max}$ finishes a passage in $F$.
\\ Define $m = k-1$ and $H_{i+1} = L F$. The following conditions hold:
\begin{enumerate}
	\item Since $p_{max}$ finishes its passage in $F$, $Act(H_{i+1}) = Act(L) \setminus \{p_{max}\} = W_m$, thus $|Act(H_{i+1})| = |Act(L_m)|-1$;
	\item By Corollary \ref{claim: q-extension}, $W_m$ is an IN-set of $H_{i+1}$, thus $H_{i+1}$ is a regular execution;
	\item Each $p \in Act(H_{i+1})$ executes $\ell_{i+1}$ critical events in $L$, and thus in $H_{i+1}$;
	\item Since $p_{max}$ finishes its passage in $F$, we get $Fin(H_{i+1}) = Fin(L) \cup \{p_{max}\} = Fin(H_i) \cup \{p_{max}\}$. Therefore $|Fin(H_{i+1})| = i+1$;
	\item Each $p \in Act(H_{i+1})$ completes $i+1$ fences in $H_{i+1}$ and $mode(p,H_{i+1}) = read$.
\end{enumerate}
We are done with the regularization phase, and thus with the entire inductive step.

\subsubsection*{\hspace{5mm} Case II}
$p_{max}$ is about to execute a critical event $f$ after $L F$.
\\ Since $p_{max}$ does not finish its passage, $Act(L F) = Act(L)$ and $Fin(L F) = Fin(L) = Fin(H_i)$. $F$ contains no critical events in $L F$, thus by corollary \ref{claim: q-extension}, $W_{k-1}$ is an IN-set of $L F$. Let $u$ be the remote variable $p_{max}$ accesses in $f$. We define:

\begin{align*}
q & = \begin{cases}
	writer(u, L F) & ,writer(u, L F) \in W_{k-1}
	\\ \perp & ,otherwise
	\end{cases}
\\
q_u & = \begin{cases}
	owner(u) & ,owner(u) \in W_{k-1}
	\\ \perp & ,otherwise
	\end{cases}
\end{align*}

Denote $Q = \{q,q_u\}$ and $W_k = W_{k-1} \setminus Q$.
\begin{claim-subsection}
\label{claim:subsection}
	$|Q| \leq 1$ (where we do not count $\perp$).
\end{claim-subsection}
\begin{proof}
	Assume $|Q| = 2$, then $q,q_u \in W_{k-1}$ and $q \neq q_u$. Since $writer(u, L F) = q$ and $q \neq owner(u)$, $q$ remotely accessed $u$ in $L F$. $W_{k-1}$ is an IN-set of $L F$, thus by IN4 $q_u \notin Act(L F)$ - a contradiction.
\end{proof}

Since $Q \subseteq W_{k-1}$, by Lemma \ref{lem: remove-invisibale-processes} with $'E' \leftarrow L F$, $'INV' \leftarrow W_{k-1}$ and $'Y' \leftarrow Q$, we have: $N = (L F)^{-Q}$ is an execution, and the following hold:
\begin{enumerate}
	\item $W_k = W_{k-1} \setminus Q$ is an IN-set of $N$.
	\item $Act(N) = Act(L F) \setminus Q = W_k \cup \{p_{max}\}$, thus $|Act(N)| \geq |Act(L)| - 1$.
	\item $Fin(N) = Fin(L F) = Fin(H_i)$.
	\item Each $p \in W_k$ executes the same events in $N$ and in $L F$, thus $p$ completes $i+1$ fences in $N$ and $mode(p,N) = read$.
	\item Each $p \in Act(N)$ executes the same critical events in $N$ and in $L F$. Since $F$ contains no critical events in $L F$, each $p \in W_k$ executes $\ell_{i+1}$ critical events in $N$, and $p_{max}$ executes $\ell_{i+1}+k-1$ critical events in $N$.
	\item $p_{max}$ is about to execute a critical event $e \sim f$ after $N$.
\end{enumerate}

We extend $N$ by letting $p_{max}$ execute $e$, and denote the resulting execution $L_k = N e$.

\begin{claim-subsection} \label{claim: W_k-is-an-IN-set}
	$W_k$ is an IN-set of $L_k$.
\end{claim-subsection}

\begin{proof}

We start by proving two properties relating to variable $u$.

\begin{itemize}

\item property 1: $writer(u,N) \notin W_k$. If $writer(u, L F) \notin W_{k-1}$ then $q = \perp$, thus the events by $writer(u, L F)$ have not been removed from $N$ and we get $writer(u, N) = writer(u, L F) \notin W_k \subseteq W_{k-1}$. Otherwise, $writer(u, L F) = q \in W_{k-1}$. $W_{k-1}$ is an IN-set of $L F$, hence, by IN5, $q$ is the only process in $Act(L F)$ to access $u$ in $L F$ (otherwise $writer(u, L F) \notin W_{k-1}$, a contradiction). Therefore, after removing the events by $q \in Q$ there is no process in $W_{k-1}$ that accesses $u$ in $N$, i.e. $writer(u, N) \notin W_k \subseteq W_{k-1}$.
    
\item property 2: $owner(u,N) \notin Act(N)$. If $owner(u) \notin Act(L F)$, then $owner(u) \notin Act(N) \subseteq Act(L F)$. Otherwise, $owner(u) \in Act(L F)$. Since $p_{max}$ remotely accesses $u$ in $f$, we have $owner(u) \neq p_{max}$, thus  $owner(u) \in W_{k-1}$. From our construction, $owner(u) \in Q$, and therefore $owner(u) \notin Act(N)$.
    
\end{itemize}
	
	$W_k$ is an IN-set of $N$, thus, by the last two properties and by Lemma \ref{lem: access-visible-variable}, IN1-IN4 hold for $W_k$ in $N e = L_k$. As IN5 holds for $W_k$ in $N$, it clearly holds for any variable $v \neq u$ in $L_k$. Consider now variable $u$. Either $e$ does not commit a write to $u$, and thus $writer(u,L_k) = writer(u,N) \notin W_k$, or $e$ is a commit write to $u$, and thus $writer(u,L_k) = p_{max} \notin W_k$. In both cases, IN5 holds for $u$ in $L_k$. As a result, $W_k$ is an IN-set of $L_k$.
\end{proof}

We now prove that $L_k$ satisfies all the conditions of Lemma \ref{regularization-phase-lemma}:
\begin{enumerate} [(1)]
	\item $e$ is not a transition event, thus $Act(L_k) = Act(N) = W_k \cup \{p_{max}\}$ (where $p_{max} \notin W_k$);
	\item By claim \ref{claim: W_k-is-an-IN-set} $W_k$ is an IN-set of $L_k$;
	\item $p_{max}$ eamxecutes $\ell_{i+1}+k-1$ critical events in $N$, and $e$ is a critical event in $N e$. Therefore $p_{max}$ executes $\ell_{i+1}+k$ critical events in $L_k$;
	\item Each $p \in W_k$ executes $\ell_{i+1}$ critical events in $L_k$;
	\item Each $p \in W_k$ executes the same events in $L_k$ and in $N$, thus $p$ completed $i+1$ fences in $L_k$, and $mode(p,L_k) = read$;
	\item $Fin(L_k) = Fin(N) = Fin(H_i)$;
	\item $|Act(L_k)| = |Act(N)| \geq |Act(L)|-1$.
\end{enumerate}

\begin{claim-section} \label{claim: regularization-upper-bound}
	The number of steps in the regularization phase is bounded by $f(i+1)$. \\ (In other words, $m \leq f(i+1)$.)
\end{claim-section}
\begin{proof}
	Assume towards a contradiction that during the regularization phase we build an execution $L_k$ such that $k > f(i+1)$. Then $L_k$ satisfies:
	\begin{itemize}
		\item $Act(L_k)$ can be written as $W_k \cup \{p_{max}\}$ (where $p_{max} \notin W_k$);
		\item $W_k$ is an IN-set of $L_k$;
		\item $Fin(L_k) = Fin(H_i)$, thus $|Fin(L_k)| = i$.
		\item $p_{max}$ executes $\ell_{i+1}+k$ critical events in $L_k$.
	\end{itemize}
	Using Lemma \ref{lem: remove-invisibale-processes} with $'E' \leftarrow L_k$ and $'INV','Y' \leftarrow W_k$, we have an execution $L'_k = L_k^{-W_k}$ such that: $Act(L'_k) = Act(L_k) \setminus W_k = \{p_{max}\}$ and $Fin(L'_k) = Fin(L_k) = Fin(H_i)$; $p_{max}$ executes the same critical events in $L'_k$ and in $L_k$, thus $p_{max}$ executes $\ell_{i+1}+k$ critical events in $L'_k$.
	\\ Hence, at most $i+1$ processes issue events in $L'_k$, i.e. the total contention of $L'_k$ is at most $i+1$. However, $p_{max}$ executed $\ell_{i+1}+k > f(i+1)$ critical events during a single passage in $L'_k$, contradicting our assumption that the algorithm is $f$-adaptive.
\end{proof}




\subsection{Construction Bounds}
We now present an analysis for the size of $Act(H_i)$ based on the upper bounds on the number of steps in for each phase. We will prove a lower bound under some restriction on the growth rate of the adaptivity function $f$.

\begin{theorem} \label{theorem: Act-lower-bound}
	Let $i \in \mathbb{N}$ be such that $f(i) \leq \dfrac{N^{2^{-f(i)}}} {f(i)! \cdot 4^{f(i)+2i}}$. Then the following lower bound holds:
	$$|Act(H_i)| \geq \frac{N^{2^{-\ell_i}}}{\ell_i! \cdot 4^{\ell_i+2i}}$$
\end{theorem}



\begin{proof}
	$ $ \newline
We assume WLOG that the adaptivity function $f$ is non-decreasing. We prove the theorem by induction on $i$. For $i=0$, we have $|Act(H_0)| \geq N$ which is trivially true.

Let $i+1$ be such that:
 $$f(i+1) \leq \dfrac{N^{2^{-f(i+1)}}} {f(i+1)! \cdot 4^{f(i+1) + 2(i+1)}}$$
 Since $f$ is non-decreasing:
\begin{align*}
	& f(i) \leq f(i+1) \leq \\
	& \qquad \dfrac{N^{2^{-f(i+1)}}} {f(i+1)! \cdot 4^{f(i+1)+2(i+1)}} \leq \dfrac{N^{2^{-f(i)}}} {f(i)! \cdot 4^{f(i)+2i}}.
\end{align*}
Hence $i$ satisfies the condition in Theorem \ref{theorem: Act-lower-bound}, and by the induction hypothesis, $|Act(H_i)| \geq \frac{N^{2^{-\ell_i}}}{\ell_i! \cdot 4^{\ell_i+2i}}$.
	
The induction step is partitioned into several sub-steps, corresponding to the phases in the construction of $H_{i+1}$ from $H_i$. In each sub-step, we establish a lower bound on the number of active processes in the intermediate executions during the respective phase, based on the lower bound established for the phases preceding it.
	
	Read phase: $|Act(G_k)| \geq \dfrac{N^{2^{-(\ell_i+k)}}}{(\ell_i+k)! \cdot 4^{\ell_i+k+2i}}$.
	\\ By induction on $k$.
	\\ Base case $k=0$: then $G_0 = H_i$ and the claim holds.
	\\ Induction step: assume we proved the claim for $k-1$. By condition (5) of Lemma \ref{read-phase-lemma}:
	\begin{align*}
	& |Act(G_k)| \geq \dfrac{|Act(G_{k-1})|-1}{10} \geq \\
	& \qquad \dfrac{\frac{N^{2^{-(\ell_i+k-1)}}}{(\ell_i+k-1)! \cdot 4^{\ell_i+k-1+2i}}-1}{10} \geq \dfrac{N^{2^{-(\ell_i+k)}}}{(\ell_i+k)! \cdot 4^{\ell_i+k+2i}}
	\end{align*}
	where the last inequality holds as long as $\ell_i+k \geq 3$, which may be assumed since $\ell_i$ increases from phase
    to phase and $k$ increases in the course of the read phase.
	
	Write phase: $|Act(J_k)| \geq \dfrac{N^{2^{-(\ell_i+s+k)}}}{(\ell_i+s+k)! \cdot 4^{\ell_i+s+k+2i} \cdot 2}$
	\\ By induction on $k$.
	\\ Base case $k=0$: 
	\\ $|Act(J_0)| \geq \dfrac{|Act(G_s)|}{2} \geq \dfrac{N^{2^{-(\ell_i+s)}}}{(\ell_i+s)! \cdot 4^{\ell_i+s+2i} \cdot 2}$.
	\\ Induction step: assume we proved the claim for $k-1$. By condition (5) of Lemma \ref{write-phase-lemma}:
	\begin{align*}
	& |Act(J_k)| \geq \dfrac {\sqrt{|Act(J_{k-1})|}} {4(\delta+k)} \geq \\
	& \quad \dfrac {\sqrt{\dfrac{N^{2^{-(\ell_i+s+k-1)}}} {(\ell_i+s+k-1)! \cdot 4^{\ell_i+s+k-1+2i} \cdot 2}}} {4(\ell_i+s+k)} \geq \\
	& \qquad \dfrac {\dfrac{\sqrt{N^{2^{-(\ell_i+s+k-1)}}}} {(\ell_i+s+k-1)! \cdot 4^{\ell_i+s+k-1+2i} \cdot 2}} {4(\ell_i+s+k)} = \\
	& \qquad \quad \dfrac{N^{2^{-(\ell_i+s+k)}}}{(\ell_i+s+k)! \cdot 4^{\ell_i+s+k+2i} \cdot 2}
	\end{align*}
	
	Regularization phase:	
	 $$|Act(L_k)| \geq \dfrac{N^{2^{-\ell_{i+1}}}}{\ell_{i+1}! \cdot 4^{\ell_{i+1}+2i+1}} - k$$
	\\ By induction on $k$.
	\\ Base case $k=0$:
	\begin{align*}
	& |Act(L_0)| \geq \dfrac{|Act(J_t)|}{2} \geq \\
	& \quad \dfrac{\dfrac{N^{2^{-(\ell_i+s+t)}}}{(\ell_i+s+t)! \cdot 4^{\ell_i+s+t+2i} \cdot 2}}{2} = \dfrac{N^{2^{-\ell_{i+1}}}}{\ell_{i+1}! \cdot 4^{\ell_{i+1}+2i+1}}
	\end{align*}
	\\ Induction step: assume we proved the claim for $k-1$. By condition (7) of Lemma \ref{regularization-phase-lemma}:
	$$|Act(L_k)| \geq |Act(L_{k-1})|-1 \geq \dfrac{N^{2^{-\ell_{i+1}}}}{\ell_{i+1}! \cdot 4^{\ell_{i+1}+2i+1}} - k$$.
	
	Therefore we have:
	\begin{equation} \label{eq: Act(H_i)-bound}
		\begin{aligned}
			& |Act(H_{i+1})| = |Act(L_m)| - 1 \geq \\
			& \qquad \dfrac{N^{2^{-\ell_{i+1}}}} {\ell_{i+1}! \cdot 4^{\ell_{i+1}+2i+1}}-(m+1)
		\end{aligned}
	\end{equation}
	
	From claim \ref{claim: regularization-upper-bound} and our assumption,
	$$m \leq f(i+1) \leq \dfrac{N^{2^{-f(i+1)}}} {f(i+1)! \cdot 4^{f(i+1) + 2(i+1)}}$$
	\\ By Claim \ref{claim: write-upper-bound}, $\ell_{i+1} \leq f(i+1)$. Therefore, we can replace $f(i+1)$ with $\ell_{i+1}$ to get:
	\begin{equation} \label{eq: m-bound}
		\begin{aligned}
			& m \leq \dfrac{N^{2^{-f(i+1)}}} {f(i+1)! \cdot 4^{f(i+1) + 2(i+1)}} \leq \\
			& \qquad \dfrac{N^{2^{-\ell_{i+1}}}} {\ell_{i+1}! \cdot 4^{\ell_{i+1} + 2(i+1)}} = \qquad \dfrac{1}{4} \cdot \dfrac{N^{2^{-\ell_{i+1}}}} {\ell_{i+1}! \cdot 4^{\ell_{i+1}+2i+1}}
		\end{aligned}
	\end{equation}
	
	
	Plugging Inequality \ref{eq: m-bound} into Inequality \ref{eq: Act(H_i)-bound} yields the required lower bound:
	\begin{align*}
		& |Act(H_{i+1})| \geq
		\dfrac{N^{2^{-\ell_{i+1}}}} {\ell_{i+1}! \cdot 4^{\ell_{i+1}+2i+1}}-(m+1) \geq \\
		& \quad \dfrac{N^{2^{-\ell_{i+1}}}} {\ell_{i+1}! \cdot 4^{\ell_{i+1}+2i+1}}-\dfrac{1}{2} \cdot \dfrac{N^{2^{-\ell_{i+1}}}} {\ell_{i+1}! \cdot 4^{\ell_{i+1}+2i+1}} = \\
		& \qquad \dfrac{1}{2} \cdot \dfrac{N^{2^{-\ell_{i+1}}}} {\ell_{i+1}! \cdot 4^{\ell_{i+1}+2i+1}} \geq
		\dfrac{N^{2^{-\ell_{i+1}}}} {\ell_{i+1}! \cdot 4^{\ell_{i+1}+2(i+1)}}
	\end{align*}
\end{proof}

We can now prove our main result.

\begin{theorem-repeat} {main-theorem}
	Let $\mathcal{A}$ be an $N$-process weak obstruction-free $f$-adaptive implementation of a mutual-exclusion lock and let $i \in \mathbb{N}$ be such that $f(i) \leq \dfrac{N^{2^{-f(i)}}} {f(i)! \cdot 4^{f(i)+2i}}$.
	Then there exists an execution $H$ whose total contention is $i+1$ and a process $p$ such that $p$ executes $i$ fences in $H$ during a single passage of its CS.\end{theorem-repeat}

\begin{proof}
Since $l_i < f(i)$, it follows from Theorem \ref{theorem: Act-lower-bound} that $|Act(H_i)| \geq f(i) \geq 1$. This implies that our construction results in an execution $H_i$, in which there is a process $p \in Act(H_i)$ and, from the properties of $H_i$, $p$ is in a middle of a passage in which it executed (and completed) $i$ fences.
Moreover, from Lemma \ref{lem: remove-invisibale-processes}, we are able to erase all active processes but $p$ from $H_i$ and obtain an execution $H$, in which $p$ executes $i$ fences in the course of a single passage, and the total contention of $H$ is $i+1$, that is, the number of fences $p$ executes is linear in the total contention of the execution.
\end{proof}


